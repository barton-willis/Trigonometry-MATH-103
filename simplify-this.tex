\documentclass[12pt,fleqn]{article}

\usepackage{xcolor}
\usepackage{fleqn}
\usepackage[UKenglish]{isodate}% http://ctan.org/pkg/isodate
\usepackage[letterpaper,paperwidth=8.5in,paperheight=11in,margin=0.5in]{geometry} 
\usepackage[USenglish]{babel}
\usepackage{hyperref}
\usepackage[activate={true,nocompatibility},final,tracking=true,kerning=true,spacing=true,factor=1100,stretch=10,shrink=10]{microtype}

\pagestyle{empty}
\usepackage{multirow}
\usepackage[T1]{fontenc} 
\usepackage{fourier}

\usepackage{enumerate,isomath,hyperref}
\usepackage{upgreek,comment}
\usepackage{graphicx}
%\usepackage[super]{nth}
\usepackage{amsmath}

\newenvironment{alphalist}{
  \begin{enumerate}[(a)]
    \addtolength{\itemsep}{-0.5\itemsep}}
  {\end{enumerate}}
  \cleanlookdateon% Remove ordinal day reference
  \newcommand{\RomanNumeralCaps}[1]
      {\MakeUppercase{\romannumeral #1}}


      \usepackage{amstext} % for \text macro
      \usepackage{array}   % for \newcolumntype macro
      \newcolumntype{L}{>{$}l<{$}} % math-mode version of "l" column type
      
      \newcommand{\dom}{\mathrm{dom}} 
      \newcommand{\range}{\mathrm{range}} 
      \newcommand{\zero}{\mathrm{zero}} 
      \newcommand{\reals}{\mathbf{R}} 
      \newcommand{\integers}{\mathbf{Z}} 
      \newcommand{\ssep}{\mid}
      \newcommand{\arcsec}{\mathrm{arcsec}}
      \newcommand{\arccsc}{\mathrm{arccsc}}
      \newcommand{\arccot}{\mathrm{arccot}}
      
      
      \title{How do you want me to simplify this?}
\begin{document}

\maketitle
\begin{quote}
\emph{My (admittedly perverse) answer is that ``to simplify''  means to write an equivalent expression that the instructor/marker likely wants or expects as an answer. It is an exercise in mind-reading.}  \\  \phantom{xxxxx} \hfill   {\mbox{\sc B.\ S.\ Thomson}}
\end{quote}
\normalsize 

Professor Thomson's goes on to  illustrates his point by asking the question, which is simpler \( \frac{532672}{1000000} \) or 
the equivalent, but reduced,  fraction \(\frac {8323}{15625} \)?  If needed, we can immediately convert the first fraction to a
decimal form, but converting the second is more work.  Our base-ten minds might say that \(\frac{532672}{1000000} \) is the simplest, but likely you've had a math teacher who would have counted the answer \( \frac{532672}{1000000} \) as wrong.  

For professor Thomson's insightful answer to the question what does simplification mean, see \url{https://www.quora.com/What-does-it-mean-to-simplify-an-expression?share=1}. These examples, plus many similar ones, illustrate that simplification is context dependent. For the context of your homework and earning the grade that you deserve, as Professor Thomson says,  is sadly is  matter of mind-reading.  On-line homework systems, with their overly strict ways of deciding correctness have made the question of simplification more important and have extended mind-reading to guessing about the algorithm used by the automated homework system.

\subsection*{Accuracy} 

\begin{equation}
    T  = \frac{1}{2} m v^2
\end{equation}

\[
    T =  m c^2 \left (\frac{1}{\sqrt{1- v^2/c^2}} - 1\right)
\]



\end{document}