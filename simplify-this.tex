\documentclass[12pt,fleqn]{article}
\usepackage{array}
\usepackage{xcolor}
\usepackage{fleqn}
\usepackage[USenglish]{isodate}% http://ctan.org/pkg/isodate
\usepackage[letterpaper,paperwidth=8.5in,paperheight=11in,margin=0.75in]{geometry} 
\usepackage[USenglish]{babel}
\usepackage{hyperref}
\usepackage[activate={true,nocompatibility},final,tracking=true,kerning=true,spacing=true,factor=1100,stretch=10,shrink=10]{microtype}
\usepackage{tcolorbox}
\usepackage{multirow}
\usepackage[T1]{fontenc} 
\usepackage{fourier}

\usepackage{enumerate,isomath,hyperref}
\usepackage{upgreek,comment}

\usepackage{graphicx}
%\usepackage[super]{nth}
\usepackage{amsmath}

\newenvironment{alphalist}{
  \begin{enumerate}[(a)]
    \addtolength{\itemsep}{-0.5\itemsep}}
  {\end{enumerate}}
  \cleanlookdateon% Remove ordinal day reference
  \newcommand{\RomanNumeralCaps}[1]
      {\MakeUppercase{\romannumeral #1}}


      \usepackage{amstext} % for \text macro
      \usepackage{array}   % for \newcolumntype macro
      \newcolumntype{L}{>{$}l<{$}} % math-mode version of "l" column type
      
      \newcommand{\dom}{\mathrm{dom}} 
      \newcommand{\range}{\mathrm{range}} 
      \newcommand{\zero}{\mathrm{zero}} 
      \newcommand{\reals}{\mathbf{R}} 
      \newcommand{\integers}{\mathbf{Z}} 
       \newcommand{\rationals}{\mathbf{Q}} 
      \newcommand{\ssep}{\mid}
      \newcommand{\arcsec}{\mathrm{arcsec}}
      \newcommand{\arccsc}{\mathrm{arccsc}}
      \newcommand{\arccot}{\mathrm{arccot}}
  


\newcommand\showdiv[1]{\overline{\smash{)}#1}}
      
      \title{How do you want me to simplify this?}
\begin{document}

\maketitle
\begin{quote}
\emph{My (admittedly perverse) answer is that ``to simplify''  means to write an 
equivalent expression that the instructor/marker likely wants or expects as an answer. It is an exercise in mind-reading.} \\  \phantom{xxxxx} \hfill   {\mbox{\sc B.\ S.\ Thomson}}
\end{quote}


\subsection*{Quick guide to simplifying}

In the following, $X$ matches any subexpression of the expression
we are simplifying. The notation $X \to Y$  means to replace
the subexpression $X$ by $Y$.

\begin{tcolorbox}
   \begin{alphalist}
   
   \item  Reduce all rational numbers to lowest terms.
   
   \item All arithmetic sums, products, and exponents of numbers should be done.
   
   \item All common additive and multiplicative terms should be combined.
   
   \item Apply identities  $1 \times X  \to  X,  0 \times  X \to  0, 1^X \to  1$, and  $X^1  \to X$.
   
   \item Provided $X$ is  nonzero,  apply identities $\frac{X}{X} \to  1$ and $X^0 = 1$.
   
   \item Provided $X$ is  nonnegative, apply the identity $\left(X^a\right)^b \to  X^{a b}$.
   
   \item Use the values of the trigonometric functions at the integer multiplies of $\uppi/6$ and $\uppi/4$ to simplify these values.
   
   \item For any odd function $O$, replace $O(x) + O(-x)$ by zero. 
   
   \item For any even function $E$, replace $E(x) - E(-x)$ by zero.
   
   \item Use the identities $\log(10^X) = X$ and $\ln(\mathrm{e}^X) = X$ 
   to replace the left side by the right side.
   
   \item For a positive integer $n$, replace $\frac{1}{\sqrt{n}}$ by $\frac{\sqrt{n}}{n}$.
   
   \item For a positive integers $m$ and $n$, replace $\sqrt{m n^2}$ by $n \sqrt{m}$. 
   An integer might need to be partially factored to put it into the form
   $m \times n^2$.

   \end{alphalist}
   \end{tcolorbox}

      

\end{document}

