\documentclass[12pt,fleqn]{article}
\usepackage{array}
\usepackage{xcolor}
\usepackage{fleqn}
\usepackage[USenglish]{isodate}% http://ctan.org/pkg/isodate
\usepackage[letterpaper,paperwidth=8.5in,paperheight=11in,margin=0.75in]{geometry} 
\usepackage[USenglish]{babel}
\usepackage{hyperref}
\usepackage[activate={true,nocompatibility},final,tracking=true,kerning=true,spacing=true,factor=1100,stretch=10,shrink=10]{microtype}
\usepackage{tcolorbox}
\usepackage{multirow}
\usepackage[T1]{fontenc} 
\usepackage[]{fourier}

\usepackage{enumerate,isomath,hyperref}
\usepackage{upgreek,comment}
\usepackage{amssymb}
\usepackage{graphicx}
%\usepackage[super]{nth}
\usepackage{amsmath}
\usepackage{multirow}
\newcounter{eq}
\setcounter{eq}{1} 


\newenvironment{alphalist}{
  \begin{enumerate}[(a)]
    \addtolength{\itemsep}{-0.5\itemsep}}
  {\end{enumerate}}
  \cleanlookdateon% Remove ordinal day reference
  \newcommand{\RomanNumeralCaps}[1]
      {\MakeUppercase{\romannumeral #1}}


      \usepackage{amstext} % for \text macro
      \usepackage{array}   % for \newcolumntype macro
      \newcolumntype{L}{>{$}l<{$}} % math-mode version of "l" column type
      
      \newcommand{\dom}{\mathrm{dom}} 
      \newcommand{\range}{\mathrm{range}} 
      \newcommand{\zero}{\mathrm{zero}} 
      \newcommand{\reals}{\mathbf{R}} 
      \newcommand{\integers}{\mathbf{Z}} 
       \newcommand{\rationals}{\mathbf{Q}} 
      \newcommand{\ssep}{\mid}
      \newcommand{\arcsec}{\mathrm{arcsec}}
      \newcommand{\arccsc}{\mathrm{arccsc}}
      \newcommand{\arccot}{\mathrm{arccot}}
  
  \newcommand{\textor}{ \mbox{ \textbf{or} }}
   \newcommand{\textand}{ \mbox{ \textbf{and} }}
      \usepackage{calc}
      \newcounter{erule}\setcounter{erule}{0}
      \newcommand{\erule}{%\
      \setcounter{erule}{\value{erule}+1}
      \textbf{\theerule\,\,}}
\newcommand\showdiv[1]{\overline{\smash{)}#1}}
      
      \title{Quick reference on solving equations}
\begin{document}

%\maketitle

\noindent The following table gives some hints on how to solve an equation for a variable \(x\). The quantities
\(X,Y,Z\), and \(W\) match with any expression that involves \(x\) and the quantities \(a\) and \(b\) match with 
any number (or constant that doesn't depend on \(x\)).  Sometimes you'll need to do some algebra (divide both sides by a
nonzero number, factor, or  other such things) to make the match.
\[
    \arraycolsep=1.4pt\def\arraystretch{2.0}
\begin{array}{|c|c|c|c|}
    \hline
\mbox{\textbf{Rule}\phantom{xx}} & \mbox{\textbf{Replace}}  & \mbox{\textbf{With}} & \mbox{\textbf{Condition(s)}}\\ \hline
%\erule & 0 x = 1    & \varnothing  & \mbox{(none)}  \\ \hline
%\erule & 0 x = 0    & \reals  & \mbox{(none)}  \\ \hline
\erule & a x = b   \phantom{xxx}   & x = \dfrac{b}{a}  & a \neq 0  \\ \hline
\erule & a x^2 + b x + c = 0  \phantom{xxx}   & x = \dfrac{-b - \sqrt{b^2 - 4 a c}}{2 a} \textor
 x = \dfrac{-b + \sqrt{b^2 - 4 a c}}{2 a}  & ( a \neq 0 ) \textand  b^2 - 4 ac \geq 0  \\ \hline
\erule & X Y = 0    &  X = 0   \textor  Y = 0  & \mbox{(none)}  \\ \hline
\erule & \dfrac{X}{Y} = 0 &  X = 0  &  Y \neq 0 \\ \hline
\erule & \dfrac{W}{X} = \dfrac{Y}{Z} &  WZ = XY & X \neq 0   \textand  Z \neq 0 \phantom{xx}\\ \hline
\erule & X^2 = Y^2 &  X = -Y \textor X = Y & \mbox{(none)}\\ \hline
\erule & X^2  = Y  & X = -\sqrt{Y} \textor X = \sqrt{Y} &  Y \geq 0 \\ \hline
\erule & |X| = Y  & X = Y \textor X = -Y & Y \geq 0 \\ \hline
\erule & \sqrt{X} = Y & X = Y^2 & Y \geq 0 \\ \hline
\erule & a^X = b \phantom{xx} & X = \dfrac{\ln(b)}{\ln(a)}  &  (0 < a < 1) \textor (1 < a) \textand (b > 0) \\ \hline
\erule & \ln(X) = a & X = \exp(a)  & \\ \hline
\erule & \ln(X) + \ln(Y) = a & X Y = \exp(a)  &  (X >0) \textand (Y > 0)\\ \hline
\erule & \cos(X) = a & X = 2 \uppi k + \cos^{-1}(a)  \textor X = 2 \uppi k - \cos^{-1}(a) 
  & -1 \leq a \leq 1 \textand k \in \integers \\ \hline
\erule & \sin(X) = a & X = 2 \uppi k + \sin^{-1}(a) \textor 
    X = 2 \uppi k + \uppi - \sin^{-1}(a) & -1 \leq a \leq 1 \textand  k \in \integers  \phantom{xx} \\ \hline
\erule & \tan(X) = a & X = 2 \uppi k + \tan^{-1}(a)   &  k \in \integers \\ \hline
  \end{array}
\]
%\end{tcolorbox}


\end{document}

